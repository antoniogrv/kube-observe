% self defined colors for colored code (can add more colors)
\definecolor{green}{rgb}{0.1,0.5,0.2} % rgb defined between 0 and 1
\definecolor{blue}{rgb}{0,0.3,0.7}
\definecolor{purple}{rgb}{0.7,0,0.7}
\definecolor{gray}{rgb}{0.5,0.5,0.5}
\definecolor{codeorange}{rgb}{0.9,0.3,0}

% writing code in the document as list
% can be changed, add changed version before your code (possible to have multiple versions within one doc)
\lstset{ 
  basicstyle=\ttfamily\UseRawInputEncoding\small,
  extendedchars=true, % lets you use non-ASCII characters; for 8-bits encodings only, does not work with UTF-8
  breaklines=true,                 
  escapeinside={\%*}{*)}, % if you want to add LaTeX within your code
  keepspaces=true,
  morekeywords={*,...}, % if you want to add more keywords
  showstringspaces=false,          
  tabsize=2, % sets default tabsize to 2 spaces
  numbersep=5pt, % how far the line-numbers are from the code
  numbers=left,                   
  numberstyle=\tiny\color{gray},
  stringstyle=\color{purple},
  commentstyle=\color{gray},
  keywordstyle=\color{blue},
  identifierstyle=\color{codeorange}
}

%=================================================================
%                           Start Document
%=================================================================
\section{Introduzione e motivazioni progettuali}
\lhead{Introduzione e motivazioni progettuali} % section header

Questo documento, intitolato \textbf{Kubernetes Observability \& Detection Engineering}, ha l'obiettivo di illustrare un lavoro progettuale di natura parzialmente compilativa e parzialmente sperimentale (i.e. con una limitata componente pratica), descrivendo in maniera esaustiva e organica l'analisi svolta nel contesto in esame, e mostrando in maniera dettagliata e rigorosa i procedimenti svolti per ottenere l'applicativo realizzato.

Nella fattispecie, il progetto è relativo alla tipologia quindici (15) fra quelle selezionabili per il corso di Penetration Testing \& Ethical Hacking, "\textit{Sistemi SOAR (Security	Orchestration, Automation and Response) e SIEM (Security Information and	Event Management)}". I temi trattati sono laterali rispetto ai contenuti principali erogati durante il corso, ma sono in ogni caso fortemente legati ad argomenti di sicurezza, e che comunque rientrano nella traccia sopracitata.

Inoltre, poiché non è stato fornito un template predefinato da compilare, la stesura di questo documento (nella misura del suo formato e dei suoi contenuti) è a discrizione dell'autore, auspicando che risulti essere completa e soddisfacente.

La scelta di questa specifica traccia è dovuta ad un interesse personale dello studente nei confronti delle tecnologie in esame, specificamente \textbf{Kubernetes}, e il desiderio di esplorare nuovi anfratti del suo ecosistema, intersezionandosi ove possibile con i temi della traccia, per quanto le possibilità di lacciarsi a specifici tool o software studiati durante il corso sia limitato. Il lettore, dunque, consideri questa attività progettuale come un'analisi - o meglio, un'introduzione - ad alcuni angoli di un vasto strumento che, ormai, non è più considerabile "\textit{secondario}", o "\textit{ausiliare}"; e che, anzi, risulta essere il principale coltellino svizzero di un qualsiasi \textit{DevSecOps Engineer} contemporaneo.

Kubernetes è spesso considerata una "\textit{bestia}", uno strumento "\textit{difficile}", "\textit{capzioso}", profondamente ramificato nel mondo cloud-native e le cui articolazioni di produzione non possono esimersi dallo svolgere le dovute considerazioni di sicurezza, come previsto in qualsiasi sistema degno di una tale popolarità. Qualsiasi utente di Kubernetes deve fare i conti con la sicurezza dei propri container e del proprio cluster, a meno di applicativi estremamente semplici e poco affascinanti; capire come questa sicurezza può essere espletata, anche nella misura tale per cui ci si limita a guardare piuttosto che agire, è vitale alla compliance dovuta agli stakeholders e agli utenti del sistema.

Il corso di Penetration Testing insegna allo studente a scrutare una macchina virtuale, smembrarla, farla propria. Questo progetto, invece, punta ad illustrare \textit{come capire che tutto questo sta accadendo}, e quali sono gli strumenti più adeguati per farlo; prima, però, viene descritto il funzionamento di un tipico cluster Kubernetes, mettendolo anche in correlazione agli ambienti didattici di virtualizzazione implementati durante il corso; vengono, poi, fornite opportunamente alcune definizioni critiche per il contesto affrontato (fra cui SIEM e SOAR) e, infine, viene proposta una piccola sperimentazione esemplificativa.

In particolare, la sperimentazione riguarda il bootstrap di un sistema SIEM su Kubernetes usando strumenti allo stato dell'arte, seguito da una serie di considerazioni di natura genuinamente pratica e tecnica sul perché il local development di sistemi di observability sia poco pratico, e su perché quindi la sperimentazione svolta non abbia davvero realizzato a pieno un SIEM, limitandosi alle fasi di bootstrapping. Verrà, infine, presentata una piccola correlazione ad alcuni sistemi reali implementati dal GARR.

Tutto ciò afferisce, in parte o in tutto, ad una branca del DevOps Engineering e della Cybersecurity spesso indicata come "Detection Engineering", in cui si pone particolare accento sulla questione dell'osservabilità ("observability"). L'esigenza che un enorme cluster composto da centinaia di macchine virtuali e ospitante migliaia di applicativi debba essere osservabile (in una misura o nell'altra) potrebbe risultare scontata; tuttavia, è facile dimenticare che il prezzo delle risorse (umane, economiche, computazionali) impiegate per realizzare sistemi di observability viene ripagato esclusivamente nei casi peggiori in cui questi sistemi arrivano a notificare disastri ed incidenti; fino ad allora, sembreranno non costituire alcun valore aggiunto. Ebbene, un incidente di sicurezza può davvero essere fatale per l'integrità e la confidenzialità dei dati, e per la disponibilità degli applicativi: fare in modo che questi ultimi siao costantemnete sotto l'occhio vigile di Kubernetes, a cui ne sarà delegata la governance e la mise-en-place, è il minimo che possiamo fare.

L'autore ha anche realizzato una tesi magistrale su Kubernetes, specializzando il proprio studio sull'orchestrazione di carichi di lavoro di Machine Learning (rif. \href{https://github.com/antoniogrv/kube-gf}{Governance MLOps: Orchestrazione sicura di carichi di Machine Learning con Kubernetes}, relatori: proff. Christiancarmine Esposito, Rocco Zaccagnino). Quindi, in termini strettamente personali, questo progetto è da considerarsi come un tentativo di colmare le lacune emerse durante tale lavoro di tesi, e riguardanti appunto l'osservabilità dei sistemi, che si intersezionano comodamente coi meccanismi SIEM e SOAR che andremo ad analizzare.

Tutti i riferimenti ("\textit{rif.}") sono raccolti nella bibliografia.