% self defined colors for colored code (can add more colors)
\definecolor{green}{rgb}{0.1,0.5,0.2} % rgb defined between 0 and 1
\definecolor{blue}{rgb}{0,0.3,0.7}
\definecolor{purple}{rgb}{0.7,0,0.7}
\definecolor{gray}{rgb}{0.5,0.5,0.5}
\definecolor{codeorange}{rgb}{0.9,0.3,0}

% writing code in the document as list
% can be changed, add changed version before your code (possible to have multiple versions within one doc)
\lstset{ 
  basicstyle=\ttfamily\UseRawInputEncoding\small,
  extendedchars=true, % lets you use non-ASCII characters; for 8-bits encodings only, does not work with UTF-8
  breaklines=true,                 
  escapeinside={\%*}{*)}, % if you want to add LaTeX within your code
  keepspaces=true,
  morekeywords={*,...}, % if you want to add more keywords
  showstringspaces=false,          
  tabsize=2, % sets default tabsize to 2 spaces
  numbersep=5pt, % how far the line-numbers are from the code
  numbers=left,                   
  numberstyle=\tiny\color{gray},
  stringstyle=\color{purple},
  commentstyle=\color{gray},
  keywordstyle=\color{blue},
  identifierstyle=\color{codeorange}
}

%=================================================================
%                           Start Document
%=================================================================
\section{Considerazioni conclusive}
\lhead{Considerazioni conclusive}

In termini compilativi, questo progetto ha permesso di esplorare più a fondo Kubernetes, i concetti dietro ai sistemi SIEM e SOAR, così come le possibili integrazioni fra Kubernetes ed Elastic (come collante per meccanismi di observability). In termini sperimentali, è stato condotto il bootstrap di un sistema SIEM (i.e. Elastic, Kibana) su Kubernetes, descrivendo in maniera del tutto riproducibile le operazioni svolte. 

Le fasi successive al bootstrapping di Elastic e Kibana implicherebbero agganciare un certo numero di integrazioni al sistema: in questo caso, si parlerebbe comunque di integrazioni che dipenderebbero molto dalla configurazione di rete e dalla tipologia di applicativi che si dovrebbero monitorare, e che in ogni caso sarebbero blindati dietro soluzioni a pagamento, nella maggior parte dei casi.

L'autore ha anche ipotizzato l'utilizzo di Grafana e Prometheus come "alternative pienamente gratuite ed open-source", ma sono sistemi che si discostano troppo dal concetto di "SIEM" e quindi non avrebbe avuto molto senso perseguire questa strada.

Lo studio condotto, quindi, rappresenta molte delle possibili considerazioni iniziali che un addetto alla sicurezza dovrebbe porsi prima di mettere in campo un sistema di observability su Kubernetes, a partire da "cosa si sta facendo" (monitorare), "su quale piattaforma lo si sta facendo" (Kubernetes), "quali strumenti mi servono" (SIEM e SOAR), "come avvio un sistema del genere" (bootstrap SIEM), "quali sono i prossimi passi" (acquistare le integrazioni adatte).

In conclusione, si rimarca che sulla repository del progetto sono presenti tutti gli script YAML utilizzati, e che il materiale investigato per produrre questo documento è presente nella bibliografia.